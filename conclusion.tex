

When scripts grow up, all too often they become unmaintainable.  One
important reason for this is that the design information the original
programmer had is lost, and difficult to recover.  For these scripts
to become maintainable programs, recovering that design information is
essential.  I have proposed that modular porting of untyped code to a
typed sister language makes the transition from scripts to programs
 not only possible, but easy, by facilitating this recovery process.

\section{Contributions}

To support this thesis, I have developed Typed Scheme, a typed sister
language of PLT Scheme.  Typed Scheme consists of two key pieces: a
system for sound interaction between typed and untyped modules, and a
type system that works with traditional Scheme idioms.  

Typed Scheme allows typed and untyped modules to freely interoperate,
with higher-order values able to flow in both directions.  It also
automatically generates runtime software contracts to enforce the
types of the typed portions of the program and blame the appropriate
offending party.  This integration satisfies 
a soundness theorem that proves that the
typed portions of the program are never blamed, generalizing Milner's
slogan to ``well-typed \emph{modules} don't go wrong''.

Typed Scheme includes a novel type system designed to accommodate
typical Scheme programming idioms.  This includes the idea of
occurrence typing, which allows type information from tests to be used
in the typing of branches.  This formulation of occurrence typing also
allows for a lightweight form of refinement types.  The type system
also includes a novel system for handling variable arity functions,
even those with complex relationships among their argument types.

Finally, Typed Scheme is implemented and distributed as a part of
PLT Scheme.  The implementation of Typed Scheme uses the PLT Scheme
module and macro system to implement a sound type system as a library
for an existing untyped language.  This implementation has been used
to port thousands of lines of existing code, and it is used in the
implementation of the PLT  suite of tools.  Experiments with the
implementation indicate that porting existing untyped PLT Scheme code
requires few changes to existing code.

\section{Future Work}

While Typed Scheme is useful today, much more work remains to be done.  

\subsection{Flow Analysis for Porting}

Existing flow analyses~\cite{shivers-thesis,ff:toplas99} generate
results that resemble Typed Scheme types.  While experience with soft
typing suggests that flow analysis is not appropriate for a type
system, it may well be useful for the process of porting untyped
Scheme code to Typed Scheme.  A tool using flow analysis could be run
once, generating a preliminary typing for Typed Scheme.  This could
alleviate much of the manual burden of annotating variables, as well
as allowing complex and computationally-expensive algorithms to be
used, since they would not be a constant part of the development
process.  

\subsection{Types for Complex Macros}

While Typed Scheme's strategy of examining only post-expansion code
performs well in most cases, it fails on the most complex macros.  The
PLT Scheme \scheme|class|~\cite{fff:classes} and
\scheme|unit|~\cite{ff:units} systems are implemented entirely with
macros---but the expansion of these macros is insufficient to recover
the programming discipline that they enforce.  Thus, typing programs
written using these systems must rely on an understanding of these
macros, as well as a type system that handles their unique features.  

\subsection{Beyond Scheme}

Scheme is not the only language that can benefit from modular addition
of types.  Language such as JavaScript, Python, and Ruby are likely to
benefit from the lessons of Typed Scheme.  These languages will
require their own type system, as well as affordances for their own
peculiar idioms and features.  