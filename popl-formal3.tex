\begin{schemeregion}

\section{Using Logic}
\label{sec:logic}

\logictrue


The second addition enables the type system to reason
logically about  predicates.  Let us return to
example~\ref{ex:logic}. At the point where the second
\scheme| (number? x) | test is performed, the type system must prove that if 
this expression evaluates to true,
 \scheme|y| cannot be a string.  We represent this knowledge
with a new kind of filter, $\imp{\num_{\x{}}}{\comp{\str}_{\y{}}}$.
This filter should be read as an implication, that if \scheme|x| is a
number, \scheme|y| is not a string.  The extension to the grammar of
filters and latent filters is shown in figure~\ref{fig:loggram}.

\begin{figure}
\[
  \begin{altgrammar}
         \p{} &::=&  \dots \lonly{\alt \imp{{\op{}}}{\p{}}} & \mbox{Filters} \\
         \ph{} & ::=& \dots  \lonly{\alt \imp{\oph{}}{\ph{}}}  & \mbox{Latent Filters} \\
  \end{altgrammar}
\]
\caption{Grammar Extension for Logical Reasoning}
\label{fig:loggram}
\end{figure}


Such filters are not immediately applicable to modify the type
environment, so they must be kept around until the implication can be
discharged.  To
accomplish this, we extend the type judgement with a proposition
environment $\Delta$, which is a set of filters (\p{}).  The new judgement has the form

\vspace{-4mm}
\large
\[
  \hastyeffphi{\e{}}{\t{}}{\phii{}}{\s{}}
\]
\normalsize
\noindent
For almost all of the typing rules, this new environment is passed through
unchanged.  The only exception is the {\sc T-If} rule, whose 
redefinition is provided in figure~\ref{fig:logif}.
%
There are two important changes.  First, the two halves
of the filter set of the test, $\op{+}$ and $\op{-}$, are added to the proposition environment for the
\tebranch, respectively.  Second, these new extended
environments are used to derive new propositions, $\opp{+}$ and
$\opp{-}$,
according to the
$\Delta \vdash \p{}$ relation, defined in figure~\ref{fig:proves}.
These new sets of propositions are also used in the typechecking of
the \tebranch, respectively.

\begin{figure}
\[
\logif
\]
\caption{If Rule with Logical Environment}
\label{fig:logif}
\end{figure}


\begin{figure}
\[
\inferrule[L-Env]
{\p{} \in \Delta}
{\Delta \vdash \p{}}
\qquad
\qquad
\inferrule[L-MP]
{{\Delta \vdash \op{}} \\ \Delta \vdash \imp{\op{}}{\opp{}}}
{{\Delta \vdash \opp{}}}
\]
\caption{Filter Derivation}
\label{fig:proves}
\end{figure}

Our proof system for deriving new propositions is quite simple.  The first rule is environment 
lookup, and the second is just
\emph{modus ponens}.  This straightforward system, however, is sufficient
for handling all the  Scheme idioms  described in this thesis.  Of course,
extending both the grammar of filter propositions and the set of
proof rules would add even greater expressivity, which we leave to
future work in the event that we discover idioms that need it.

\subsection{Generating Implications}

We have seen how filters using implication are used, but not
where they are generated.  Since we want to use these filters to
express the semantics of complex
\scheme|if| expressions, the point in the system to extend is the \cf
metafunction:


$$%
\begin{array}{l@{\ =\ }l}
\combfilter{(\op{1_+}|\op{1_-})} {(\op{2_+}|\op{2_-})} {\bot|\empt} & \\
\multicolumn{2}{r}{\hspace{.5in}\op{1_+},\op{2_+}|
\logic{(\imp{\op{1_+}}{\op{2_-}}),(\imp{\op{2_+}}{\op{1_-}})}{\empt}}
\end{array}
$$
\noindent
This pattern corresponds to the encoding of \scheme|and| using
conditionals.  In particular, \scheme|(and A B)| is encoded as
\scheme|(if A B #f)|.  Therefore, the \cf rule considers cases where
the third argument corresponds to false.  
Thus, instead of having no information if an \scheme|and|
expression is false, we now have two implications: $\imp{\op{1_+}}{\op{2_-}}$ and
  $\imp{\op{2_+}}{\op{1_-}}$.  

Of course, other minor changes are required in the other metafunctions
to abstract and apply filters with implication.  We omit the laborious details.

\subsection{A Logical Example}

\newcommand{\hast}[2]{#1 $:$ #2}
\def\gzero{\hast{\scheme|input|}{\scheme|(U S N)|},\ {\hast{\scheme|extra|}{\scheme|(Pair A A)|}}}
\def\gone{\hast{\scheme|input|}{\scheme|N|},\ {\hast{\scheme|extra|}{\scheme|(Pair N A)|}}}
\def\gtwo{\hast{\scheme|input|}{\scheme|S|},\ {\hast{\scheme|extra|}{\scheme|(Pair N A)|}}}

\def\gzeroname{$\Gamma_0$}
\def\gonename{$\Gamma_1$}
\def\gtwoname{$\Gamma_1$}

\def\done{\ma{\num_{\scheme|input|},\num_{\pecar(\scheme|extra|)}}}
\def\dtwo{\ma{\num_{\pecar(\scheme|extra|)} \supset
  {\comp{\num}_{\scheme|input|}}, {\comp{\num}_{\scheme|input|}} \supset \num_{\pecar(\scheme|extra|)}}}

\def\dzeroname{$\Delta_0$}
\def\donename{$\Delta_1$}
\def\dtwoname{$\Delta_2$}
\def\dthreename{$\Delta_3$}

\def\dthree{$\num_{\scheme|input|}$}
\def\dfour{$\comp{\num}_{\scheme|input|}$}

\def\nothing{$; \empt|\empt ; \noeffect$}


\begin{schemeregion}

\begin{figure}
\schemeinput{expanded-cond.ss}
\caption{Expansion of Example~\ref{ex:full}}
\label{fig:ex:full-expanded}
\end{figure}



\begin{figure}[t!]
\mbox{\empt$;$\gzeroname}\ $\vdash$ \mbox{\scheme|(if (if (number?
  input) ...) ...)|} $: \num$ \nothing
\begin{list}{$\bullet$}{\leftmargin=2em}
%\setlength{\listparindent}{0pt}
%\addtolength{\itemsep}{-0.5\baselineskip}
\renewcommand{\labelitemi}{$\bullet$}
\renewcommand{\labelitemii}{$\bullet$}
\renewcommand{\labelitemiii}{}
\renewcommand{\labelitemiv}{$\bullet$}

\item \mbox{\empt$;$\gzeroname}\ $\vdash$ \mbox{\scheme|(if (number? input) ...)|} 
$ : \bool\ 
;\ 
$\donename$ | $\dtwoname$
\ ;\  \noeffect$ 
\begin{list}{$\bullet$}{\leftmargin=1em}
\item \mbox{\empt$;$\gzeroname}\ $\vdash$ 
\mbox{\scheme|(number? input)|} $: \bool \ ;$
\mbox{$\num_{\scheme|input|}|\comp{\num}_{\scheme|input|}$} $;
\noeffect$
\begin{list}{$\bullet$}{\leftmargin=1em}
\item \dots
\end{list}
{\raggedright
\item \dthree$;$\gzeroname $\vdash$ \scheme|(number? (car extra))| $: \bool ;$$\Delta_3|\Delta_4 ; \noeffect$
}
\begin{list}{$\bullet$}{\leftmargin=1em}
\item \dots
\end{list}

\item \dfour$;$\gzeroname $\vdash$ \scheme|#f| $: \fft ; \bot|\empt ; \noeffect$
\begin{list}{$\bullet$}{\leftmargin=1em}
\item \dots
\end{list}

\end{list}
\item \donename$;$\gone\ $\vdash$ \scheme|(+ input (car extra))| $ : $
  \scheme|N| \nothing

  \begin{list}{$\bullet$}{\leftmargin=1em}
\renewcommand{\labelitemii}{}
  \item \dots
  \end{list}
\item 
\mbox{\dtwoname}$;$ \mbox{\gzeroname}\ $\vdash$ 
\mbox{\scheme|(if  (number? (car extra)) ...)|} $ : $ \scheme|N|
\nothing 
\begin{list}{$\bullet$}{\leftmargin=1em}
\item \dtwoname$;$\gzeroname $\vdash$ \scheme|(number? (car extra))|
$: \bool ;$
$\Delta_3|\Delta_4 ; \noeffect$

\begin{list}{$\bullet$}{\leftmargin=1em}
\item \dots
\end{list}

\item \dtwoname$,\Delta_3,\comp{\num}_{\scheme|input|};\Gamma_2\vdash$
 \scheme|(+ (string->number input) (car  extra))| $ : \num ;
 \empt|\empt ; \noeffect$

\begin{list}{$\bullet$}{\leftmargin=1em}
\item \dots
\end{list}

\item \dtwoname$,\Delta_4;\Gamma_0\vdash$ \scheme|0| $ : \num ;
 \empt|\bot ; \noeffect$
\end{list}




\vspace{2mm}
where
\vspace{2mm}


\begin{tabular}{l@{\quad=\quad}l}
\gzeroname & \gzero \\
%\gonename & \gone \\
\gtwoname &  \gtwo \vspace{5mm} \\
%\dzeroname & \empt \\
\donename & \done \\
\dtwoname & $\num_{\pecar(\scheme|extra|)} \supset
  {\comp{\num}_{\scheme|input|}}$, \\ 
\multicolumn{2}{r}{${\comp{\num}_{\scheme|input|}} \supset \num_{\pecar(\scheme|extra|)}$}  \\
$\Delta_3$ & $\num_{\pecar(\scheme|extra|)}$ \\
$\Delta_4$ & $\comp{\num}_{\pecar(\scheme|extra|)}$ \\
%$\Delta_5$ & \dtwoname$,\num_{\pecar(\scheme|extra|)},\comp{\num}_{\scheme|input|}$\\
%$\Delta_6$ & \dtwoname$,\comp{\num}_{\pecar(\scheme|extra|)}$ \\
\end{tabular}

\end{list}
\caption{Abbreviated Type Derivation for Expanded Example~\ref{ex:full}}
\label{fig:type-deriv}
\end{figure}

\end{schemeregion}


We now turn again to example~\ref{ex:full}. 
The expansion into the core forms of \lts is given in
figure~\ref{fig:ex:full-expanded}.  The full type derivation is given
in figure~\ref{fig:type-deriv}.  
%
Here the filter set for the first
test expression, expanded into an \scheme|if| expression in
figure~\ref{fig:ex:full-expanded}, in the \scheme|cond| is

\begin{tabular}{c}
$\done|$\\
$(\dtwo)$\\
\end{tabular}

\noindent
 abbreviated as $\Delta_1|\Delta_2$.  The first half is added to
the environment for the first expression, and the second half to the
environment for the remainder.  Then the filter set for the second
test expression is 
$$\num_{\pecar(\scheme|extra|)}|\comp{\num}_{\pecar(\scheme|extra|)}$$
\noindent
  Therefore,
following the {\sc T-If} and {\sc L-MP} rules, the type system can derive
$\comp{\num}_{\scheme|input|}$ from $\Delta_2$ and $\Delta_3$, and therefore \scheme|input|$:\str$. 
 The type system can thus prove that \scheme|input| must be a string
 in the right hand side of the second clause, meaning that the call to
 \scheme|string->number| is safe.

\end{schemeregion}
