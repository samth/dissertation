\begin{schemeregion}

%\section{A Variable-Arity Type System}
\label{sec:formal}

\definecolor{lightgray}{rgb}{.90,.90,.90}
\definecolor{gray}{rgb}{.50,.50,.50}
\definecolor{mediumgray}{rgb}{.60,.60,.60}
\definecolor{darkgray}{rgb}{.25,.25,.25}

\newcommand{\stain}[1]{\colorbox{lightgray}{{\ma{#1}}}}

% types
\newmeta\t{\tau}

% terms
\newmeta\e{e}
\newmeta\v{v}

% contexts
\newmeta\E{E}

% environments
\newmeta\G{\Gamma}
\newmeta\D{\Delta}
\newmeta\S{\Sigma}

% mapping
\newmeta\X{X}

%% abstract syntax

% types
\newmeta\a{\alpha}
\newmeta\tb{\beta}

\newbfop\num{Integer}
\newbfop\bool{Boolean}
\newbfop\listof{Listof}

\newttop\rdot{.}

\newcommand\funtype[2]{\ma{(#1 \to #2)}}
\newcommand\alltype[2]{\ma{(\forall\ (#1)\ #2)}}

\newcommand\mfuntype[2]{\ma{(\overvec{#1} \to #2)}}
\newcommand\malltype[2]{\ma{(\forall\ (\overvec{#1})\  #2)}}

\newcommand\listtype[1]{\ma{(\listof\ #1)}}
\newcommand\malltypesing[2]{\ma{(\forall\ (#1)\  #2)}}
\newcommand\mfuntypebi[3]{\ma{(#1\ #2 \to #3)}}

\newcommand\dotted[1]{\ma{#1\ \textbf{...}}}
\newcommand\dalltype[4]{\ma{(\forall\ (#1\ \dotted{#2}\ #3)\ #4)}}

\newcommand\starred[1]{\ma{#1^*}}

% terms
\newmeta\x{x}
\newmeta\p{p}
\newmeta\n{n}
\newmeta\b{b}

\newttop\zerop{zero?}
\newttop\inc{inc}
\newttop\dec{dec}

\newttop\cfix{fix}
\newttop\cif{if}
\newttop\tru{\#t}
\newttop\fls{\#f}

\newttop\eqv{=}
\newttop\plus{plus}
\newttop\minus{minus}
\newttop\mult{mult}

\newttop\cinst{@}

\newttop\ccons{cons}
\newttop\cnull{null}
\newttop\ccar{car}
\newttop\ccdr{cdr}
\newttop\cnullp{null?}
\newttop\clist{list}

\newttop\capply{apply}
\newttop\clift{map}
\newttop\cany{ormap}
\newttop\call{andmap}

\newttop\cerror{error}

\newcommand\ESOPabs[3]{\ma{(\lambda\ ([#1 : #2])\ #3)}}
\newcommand\ESOPcomb[2]{\ma{(#1\ #2)}}
\newcommand\tyabs[2]{\ma{(\Lambda\ (#1)\ #2)}}
\newcommand\tcomb[2]{\ma{(\cinst\ #1\ #2)}}
\newcommand\eif[3]{\ESOPcomb{\cif}{#1\ #2\ #3}}
\newcommand\efix[1]{\ESOPcomb{\cfix}{#1}}

\newcommand\mabs[3]{\ma{(\lambda\ (\overvec{[#1 : #2]})\ #3)}}
\newcommand\mcomb[2]{\ESOPcomb{#1}{\overvec{#2}}}
\newcommand\mtabs[2]{\tyabs{\overvec{#1}}{#2}}
\newcommand\mtcomb[2]{\tcomb{#1}{\overvec{#2}}}

\newcommand\econs[3]{\ma{(\ccons_{#1}\ #2\ #3)}}
\newcommand\enull[1]{\ma{\cnull_{#1}}}
\newcommand\ecar[1]{\ESOPcomb{\ccar}{#1}}
\newcommand\ecdr[1]{\ESOPcomb{\ccdr}{#1}}
\newcommand\enullp[1]{\ESOPcomb{\cnullp}{#1}}

\newcommand\lerror[0]{\ma{\cerror_{\text{L}}}}

% typing judgments
\newcommand\tvalid[2]{\ma{#1 \vdash #2}}
\newcommand\dtvalid[2]{\ma{#1 \mathbin{\triangleright} #2}}
\newcommand\ifnotempty[3]{\if #1 \@empty #2 \else #3 \fi}
\newcommand\comma[1]{\ifnotempty{#1}{,}{}}
\newcommand\ESOPhastype[5]{\ensuremath{#1 \comma{#1} #2 \comma{#2} #3 \vdash #4 \mathbin{:} #5}}
\newcommand\hasftype[5]{\ensuremath{#1, #2, #3 \vdash #4 \mathbin{:} #5}}
\newcommand\hasfdtype[5]{\ma{#1, #2, #3 \vdash #4 \mathbin{\triangleright} #5}}

% contexts
\newcommand\hole[0]{\ma{[\,]}}

\newcommand\inctx[2]{\ma{#1[#2]}}
\newcommand\inE[1]{\inctx{\E{}}{#1}}

% reductions
\newcommand\red{\hookrightarrow}
\newcommand\reds{\red^*}

\renewcommand\overvec[2][\relax]{\vbox{\mathsurround=0pt\ialign{##\hfil&&\hfil##\crcr
  \rightarrowfill&&$\scriptstyle{#1}$\crcr
  \noalign{\nointerlineskip}$\hfil\displaystyle{#2}\hfil$&&\crcr}}}

\newcommand\set[1]{\ma{\{#1\}}}
\newcommand\setcomp[2]{\ma{[ #1 \ |\ #2 ]}}
\newcommand{\ext}[3]{\ma{#1[#2 \mapsto #3]}}
\newcommand{\mext}[3]{\ma{#1[\overvec{#2 \mapsto #3}]}}
\newcommand{\subst}[3]{\ma{#1[#2 \mapsto #3]}}
\newcommand{\msubst}[3]{\ma{#1[\overvec{#2 \mapsto #3}]}}

\newrmop\appears{appears}
\newrmop\free{free}
\newrmop\error{error}
\newrmop\dom{dom}

\newcommand\tprim[1]{\ma{\delta_{\t{}}(#1)}}

\newitop\transdots{td}

\newcommand\ttransdots[0]{\ma{\transdots_{\t{}}}}
\newcommand\td[3]{\ma{\ttransdots(#1, #2, #3)}}
\newcommand\tddef[1]{\td{#1}{\tb{}}{\overvec[m]{\tb{k}}}}

\newcommand\etransdots[0]{\ma{\transdots_{\e{}}}}
\newcommand\etd[3]{\ma{\etransdots(#1, #2, #3)}}
\newcommand\etddef[1]{\etd{#1}{\tb{}}{\overvec[m]{\tb{k}}}}

\newcommand\xtransdots[0]{\ma{\transdots_{\x{}}}}
\newcommand\xtd[4]{\ma{\xtransdots(#1, #2, #3, #4)}}
\newcommand\xtdnew[2]{\ma{\xtd{#1}{\tb{}}{\overvec[m]{\tb{k}}}{#2}}}
\newcommand\xtddef[1]{\xtdnew{#1}{\X{}}}

\newcommand\dtransdots[0]{\ma{\transdots_{d}}}
\newcommand\dtd[4]{\ma{\dtransdots(#1, #2, #3, #4)}}
\newcommand\dtddef[1]{\dtd{#1}{\tb{}}{\overvec[m]{\tb{k}}}{\X{}}}

\newitop\substdots{sd}
\newcommand\sd[4]{\ma{\substdots(#1, #2, #3, #4)}}
\newcommand\sddef[1]{\ma{\substdots(#1, \a{r}, \t{r}, \tb{})}}


The development of our formal model starts from the syntax of a multi-arity version of
System~F~\cite{girard71}, enriched with variable-arity functions.
A technical
report~\cite{dots-tr} contains the full set of type rules as well as a semantics and soundness theorem for
this model.

\section{Syntax}

We extend System F with multiple-arity functions at both the type and term
 level, lists, as well as functions that accept rest arguments. The use of
 multiple-arity functions establishes the proper problem context.  Lists and
 rest-argument functions suffice to explain how both kinds of
 variable-arity functions interact.

\begin{figure}[t]
{\small{  $$\begin{array}{rcl}
    \p{} & ::= & \eqv
           \alt  \plus
           \alt  \minus
           \alt  \mult
           \alt  \ccar
           \alt  \ccdr
           \alt  \cnullp \\
    \v{} & ::= & \n{}
           \alt  \b{}
           \alt  \p{}
           \alt  \enull{\t{}}
           \alt  \econs{\t{}}{\v{}}{\v{}}
           \alt   \mabs{\x{}}{\t{}}{\e{}}
           \alt  \mtabs{\a{}}{\e{}} \\
           & \alt & \tyabs{\overvec{\a{}}\ \dotted{\a{}}}{\e{}}
           \alt (\lambda\ (\overvec{[\x{} : \t{}]}\ \rdot\ [\x{} : \starred{\t{}}])\ \e{}) 
            \alt  (\lambda\ (\overvec{[\x{} : \t{}]}\ \rdot\ [\x{} : \dotted{\t{}}_{\a{}}])\ \e{}) \\
    \e{} & ::= & \v{}
           \alt  \x{}
           \alt  \mcomb{\e{}}{\e{}}
           \alt  \eif{\e{}}{\e{}}{\e{}} 
           \alt \econs{\t{}}{\e{}}{\e{}}
           \alt \lerror \\
%           \alt  \ESOPcomb{\capply^*}{\e{}\ \overvec{\e{}}\ \e{}} 
           & \alt & \tcomb{\e{}}{\overvec{\t{}}}
           \alt  \tcomb{\e{}}{\overvec{\t{}}\ \dotted{\t{}}_{\a{}}}
           \alt  \ESOPcomb{\capply}{\e{}\ \overvec{\e{}}\ \e{}} \\
           & \alt & \ESOPcomb{\clift}{\e{}\ \e{}} 
            \alt  \ESOPcomb{\cany}{\e{}\ \e{}}
           \alt  \ESOPcomb{\call}{\e{}\ \e{}} \\
    \t{} & ::= & \num{}
           \alt  \bool{}
           \alt  \a{}
           \alt  \listtype{\t{}} 
            \alt  \mfuntype{\t{}}{\t{}} \\
           & \alt & \funtype{\overvec{\t{}}\ \starred{\t{}}}{\t{}}
           \alt \funtype{\overvec{\t{}}\ \dotted{\t{}}_{\a{}}}{\t{}} 
           \alt   \malltype{\a{}}{\t{}} 
           \alt \alltype{\overvec{\a{}}\ \dotted{\a{}}}{\t{}} \\
  \end{array}$$
}}
\caption{Syntax\label{fig:dots-syntax}}
\end{figure}

The grammar in figure~\ref{fig:dots-syntax} specifies the abstract syntax. We use a syntax
close to that of Typed Scheme, including the use of \cinst{} to denote
type application.  The use of the vector notation \overvec{\e{}}
denotes a (possibly empty) sequence of forms (in this case,
expressions).  In the form $\overvec[n]{\e{k}}$, $n$ indicates the
length of the sequence, and the term $\e{k_i}$ is the $i$th element.
The subforms of two sequences of the same length have the same
subscript, so $\overvec[n]{\e{k}}$ and $\overvec[n]{\t{k}}$ are
identically-sized sequences of expressions and types, respectively,
whereas $\overvec[m]{\e{j}}$ is unrelated.  If all vectors are the
same size the sizes are dropped, but the subscripts remain. Otherwise the
addition of starred pre-types, dotted type variables, dotted
pre-types, and special forms is needed to operate on non-uniform
rest arguments.

A {\em starred pre-type}, which has the form $\starred{\t{}}$ (by
analogy to the Kleene star), is used in
the types of uniform variable-arity functions whose rest parameter
contains values of type $\t{}$.  It appears only  as the last element
in the domain of a function type or as the type of a uniform
rest argument.

A {\em dotted type variable}, which has the form $\dotted{\a{}}$, serves as
a placeholder in a type abstraction.  Its presence signals that the
type abstraction can be applied to an arbitrary number of types.  A
dotted type variable can  appear only as the last element in the list
of parameters to a type abstraction.  We call type abstractions that
include dotted type variables {\em dotted type abstractions}.

A {\em dotted pre-type}, which has the form $\dotted{\t{}}_{\a{}}$, is a
type that is parameterized over a dotted type variable.  When a type
instantiation associates the dotted type variable \dotted{\a{}} with a
sequence $\overvec[n]{\t{}}$ of types, the dotted pre-type
$\dotted{\t{}}_{\a{}}$ is replaced by $n$ copies of $\t{}$, where
$\a{}$ in the $i$th copy of $\t{}$ is replaced with $\t{i}$.  In the
syntax, dotted pre-types can appear only in the rightmost position of
a function type, as the type of a non-uniform rest argument, or as the
last argument to \cinst{}.

In this model the special forms \cany{}, \call{}, and \clift{} are
restricted to applications involving non-uniform rest arguments, and
\capply{} is restricted to applications involving rest arguments.
  In Typed Scheme, of course, the \scheme|map|, \scheme|ormap|,
  \scheme|andmap| and \scheme|apply| functions work on list types in
  their most general form.

\section{Type System}

The type system is an extension of the type system of System F to
handle the new linguistic constructs.  We start with the changes to
the environments and judgments, plus the major changes to
the type validity relation.  Next we present relations used for dotted
types and expressions that have dotted pre-types instead of types.
Then we discuss the changes to the standard typing relation, and
finally we discuss the metafunctions used to define the new typing
judgments.

The environments and judgments used in our type system are similar to
those used for System F except as follows:

\begin{itemize}
\item The type variable environment (\D{}) includes both dotted and
  non-dotted type variables.
\item There is a new class of environments (\S{}), which map
  non-uniform rest parameters to dotted pre-types.
\item There is also an additional validity relation
  \dtvalid{\D{}}{\dotted{\t{}}_{\a{}}} for dotted pre-types.
\item The use of \S{} makes typing relation \hasftype{\G{}}{\D{}}{\S{}}{\e{}}{\t{}}  a
  five-place relation.
\item There is an additional typing relation
  \hasfdtype{\G{}}{\D{}}{\S{}}{\e{}}{\dotted{\t{}}_{\a{}}} for
  assigning dotted pre-types to expressions.
\end{itemize}

The type validity relation checks the validity of two forms---types
and dotted type variables.  The additional rules for establishing type
validity of non-uniform variable-arity types are provided in figure~\ref{fig:tvalid}, along
with an additional relation which checks the validity of dotted pre-types.

\begin{figure}
{{  $$
    \inferrule[TE-DVar]{\dotted{\a{}} \in \D{}}{\tvalid{\D{}}{\dotted{\a{}}}}
    \qquad
    \inferrule[TE-DFun]{{\dtvalid{\D{}}{\dotted{\t{r}}_{\a{}}}} \\\\
                        {\overvec{\tvalid{\D{}}{\t{j}}}} \\                         
                        {\tvalid{\D{}}{\t{}}}}
                       {\tvalid{\D{}}{\funtype{\overvec{\t{j}}\ \dotted{\t{r}}_{\a{}}}{\t{}}}
    }
    \qquad
    \inferrule[TE-DAll]{\tvalid{\D{} \cup \set{\overvec{\a{j}}, \dotted{\tb{}}}}{\t{}}}
                       {\tvalid{\D{}}{\alltype{\overvec{\a{j}}\
                             \dotted{\tb{}}}{\t{}}}} 
$$
$$
\inferrule[TDE-Pretype]{{\tvalid{\D{}}{\dotted{\a{}}}} \\\\
                         {\tvalid{\D{} \cup \set{\a{}}}{\t{}}}}
                        {\dtvalid{\D{}}{\dotted{\t{}}_{\a{}}}}
$$}}
\caption{Type Validity rules}
\label{fig:tvalid}
\end{figure}

When validating a dotted pre-type $\dotted{\t{}}_{\a{}}$, the bound
$\a{}$ is checked to make sure that it is indeed a valid dotted type
variable.  Then $\t{}$ is checked in an environment where the bound is
allowed to appear free.  It is possible for a dotted pre-type to be
nested somewhere within a dotted pre-type over the same bound:
\begin{exmp}
\begin{schemedisplay}
(: f (All (alpha ...) 
       ((alpha dotsa -> alpha) dotsa -> (alpha dotsa -> (Listof Integer)))))
(define ((count . fs) . args)
  (map (lambda: ([x : Any]) (if x 1 0))
       (map (lambda: ([f : (alpha dotsa -> alpha)]) (apply f args))
	    fs)))
\end{schemedisplay}
\end{exmp}

To illustrate how such a type might be used, we instantiate this
sample type with the sequence of types \num \bool:

\begin{schemedisplay}
((Integer Boolean -> Integer) (Integer Boolean -> Boolean)
 -> (Integer Boolean -> (Listof Integer)))
\end{schemedisplay}
There are two functions in the domain of the type, each of which
corresponds to an element in our sequence.  All functions have the same
domain---the sequence of types; the $i$th function returns the $i$th
type in the sequence.

\begin{figure}
$$\begin{array}{c}
\inferrule[TD-Var]{\S{}(\x{}) = \dotted{\t{}}_{\a{}}}
                  {\hasfdtype{\G{}}{\D{}}{\S{}}{\x{}}{\dotted{\t{}}_{\a{}}}} 
\qquad\qquad
\inferrule[TD-Map]{{\hasfdtype{\G{}}{\D{}}{\S{}}{\e{r}}{\dotted{\t{r}}_{\a{}}}} \\\\
                    {\hasftype{\G{}}{\D{} \cup \set{\a{}}}{\S{}}{\e{f}}{\funtype{\t{r}}{\t{}}}}}
                   {\hasfdtype{\G{}}{\D{}}{\S{}}{\ESOPcomb{\clift}{\e{f}\ \e{r}}}{\dotted{\t{}}_{\a{}}}} \\
\end{array}$$
\caption{Typing Rules for Pre-types}
\label{fig:td}
\end{figure}

The rules in figure~\ref{fig:td} are the typing rules for the two forms of
expressions that have dotted pre-types.  The {\sc TD-Var} rule 
 just checks for the variable in \S{}.  The
{\sc TD-Map} rule assigns a type to a function position.  Since the
function needs to operate on each element of the sequence represented
by $\e{r}$, not on the sequence as a whole, the domain of the
function's type is the base $\t{r}$ instead of the dotted type
$\dotted{\t{r}}_{\a{}}$.  This type may include free references to the
bound $\a{}$, however.  Therefore, we must check the function in an
environment extended with $\a{}$ as a regular type variable.

As expected, most of the typing rules are simple additions of
multiple-arity type and term abstractions and lists to System F.  For
uniform variable-arity functions, the introduction rule treats the
rest parameter as a variable whose type is a list of the appropriate
type.  There is only one elimination rule, which deals with the
special form $\capply{}$; other eliminations such as direct
application to arguments are handled via the coercion rules.

The type rules in figure~\ref{fig:type-rules} concern non-uniform
variable-arity functions.  These functions also have one introduction
and one elimination rule.  The rule {\sc T-Ormap}
and its absent counterpart {\sc T-Andmap} are similar to that of {\sc
  TD-Map} in that the dotted pre-type bound of the second argument is
allowed free in the type of the first argument.  In contrast to
uniform variable-arity functions, non-uniform variable-arity functions
must be applied with the \capply function in this calculus.

While {\sc T-DTAbs}, the introduction rule for dotted type
abstractions, follows straightforwardly from the rule for normal type abstractions, the
elimination rules are different.  There are two elimination
rules: {\sc T-DTApp} and {\sc T-DTAppDots}.  The former handles type
application of a dotted type abstraction where the dotted type
variable corresponds to a sequence of types, and the latter deals with
the case when the dotted type variable corresponds to a dotted
pre-type.

\begin{figure}[t]
$$\begin{array}{c}
\inferrule[T-DAbs]{{\overvec{\tvalid{\D{}}{\t{k}}}} \\
                   {\dtvalid{\D{}}{\dotted{\t{r}}_{\a{}}}} \\
                   {\hasftype{\msubst{\G{}}{\x{k}}{\t{k}}}{\D{}}
                             {\subst{\S{}}{\x{r}}{\dotted{\t{r}}_{\a{}}}}
                             {\e{}}{\t{}}}}
                  {\hasftype{\G{}}{\D{}}{\S{}}
                            {(\lambda\ (\overvec{[\x{k} : \t{k}]} \rdot [\x{r} : \dotted{\t{r}}_{\a{}}])\ \e{})}
                            {\funtype{\overvec{\t{k}}\ \dotted{\t{r}}_{\a{}}}{\t{}}}} \\
\\
\inferrule[T-DApply]{{\hasftype{\G{}}{\D{}}{\S{}}{\e{f}}{\funtype{\overvec{\t{k}}\ \dotted{\t{r}}_{\a{}}}{\t{}}}} \\\\
                     {\overvec{\hasftype{\G{}}{\D{}}{\S{}}{\e{k}}{\t{k}}}} \\
                     {\hasfdtype{\G{}}{\D{}}{\S{}}{\e{r}}{\dotted{\t{r}}_{\a{}}}}}
                    {\hasftype{\G{}}{\D{}}{\S{}}{\ESOPcomb{\capply}{\e{f}\ \overvec{\e{k}}\ \e{r}}}{\t{}}}
\\
\inferrule[T-Ormap]{{\hasfdtype{\G{}}{\D{}}{\S{}}{\e{r}}{\dotted{\t{r}}_{\a{}}}} \\\\
                   {\hasftype{\G{}}{\D{} \cup \set{\a{}}}{\S{}}{\e{f}}{\funtype{\t{r}}{\bool}}}}
                  {\hasftype{\G{}}{\D{}}{\S{}}{\ESOPcomb{\cany}{\e{f}\ \e{r}}}{\bool}} \\
\\
\inferrule[T-DTAbs]{\hasftype{\G{}}{\D{} \cup \set{\overvec{\a{k}}, \dotted{\tb{}}}}{\S{}}{\e{}}{\t{}}}
                   {\hasftype{\G{}}{\D{}}{\S{}}
                             {\tyabs{\overvec{\a{k}}\ \dotted{\tb{}}}{\e{}}}
                             {\alltype{\overvec{\a{k}}\ \dotted{\tb{}}}{\t{}}}} \\
\\
\inferrule[T-DTApp]{{\overvec[n]{\tvalid{\D{}}{\t{j}}}} \\
                    {\overvec[m]{\tvalid{\D{}}{\t{k}}}} \\
                    {\overvec[m]{\tb{k}} \textrm{ fresh}} \\
                    {\hasftype{\G{}}{\D{}}{\S{}}{\e{}}
                              {\alltype{\overvec[n]{\a{j}}\ \dotted{\tb{}}}
                                       {\t{}}}}}
                   {\hasftype{\G{}}{\D{}}{\S{}}
                             {\tcomb{\e{}}
                                    {\overvec[n]{\t{j}}\ \overvec[m]{\t{k}}}}
                             {\tddef{\t{}[\overvec[n]{\a{j} \mapsto \t{j}}]}
                              [\overvec[m]{\tb{k} \mapsto \t{k}}]}} \\
\\
\inferrule[T-DTAppDots]{{\overvec{\tvalid{\D{}}{\t{k}}}} \\
                        {\dtvalid{\D{}}{\dotted{\t{r}}_{\tb{}}}} \\
                        {\hasftype{\G{}}{\D{}}{\S{}}{\e{}}
                                  {\alltype{\overvec{\a{k}}\ \dotted{\a{r}}}
                                  {\t{}}}}}
                       {\hasftype{\G{}}{\D{}}{\S{}}
                                 {\tcomb{\e{}}
                                        {\overvec{\t{k}}\ \dotted{\t{r}}_{\tb{}}}}
                                 {\sddef{\t{}[\overvec{\a{k} \mapsto \t{k}}]}}} 
\end{array}$$

\caption{Selected Type Rules\label{fig:type-rules}}
\end{figure}

The {\sc T-DTAppDots} rule is more straightforward, as it is just a
substitution rule.  Replacing a dotted type variable with a dotted
pre-type is more involved than normal type substitution, however,
because we need to replace the dotted type variable where it appears
as a dotted pre-type bound.  The metafunction \substdots{} performs
this substitution.  Selected cases of the definition of \substdots{}
appear in figure~\ref{fig:metafuns}; the remaining clauses perform
structural traversals.

The {\sc T-DTApp} rule must first expand out dotted pre-types that use
the dotted type variable before performing the appropriate
substitutions.  To do this, it uses the metafunction \ttransdots{} on a
sequence of fresh type variables of the appropriate length to expand
dotted pre-types that appear in the body of the abstraction's type
into a sequence of copies of their base types.  These copies are first
expanded with \ttransdots{} and then in each copy the free occurrences
of the bound are replaced with the corresponding fresh type variable.
Normal substitution is performed on the result of \ttransdots{},
mapping each fresh type variable to its corresponding type argument.
The interesting cases of the definition of \ttransdots{} also appear in
figure~\ref{fig:metafuns}.

\begin{figure}[t]
$$\begin{array}{lcl}
  \sddef{\a{r}} & = & \t{r} \\
  \sddef{\a{}} & = & \a{} \hspace*{2em}
    \text{where $\a{} \neq \a{r}$} \\
  \sddef{\funtype{\overvec{\t{j}}\ \dotted{\t{r}'}_{\a{r}}}{\t{}}} & = \\
  \multicolumn{3}{l}{\hspace*{2em}
    \funtype{\overvec{\sddef{\t{j}}}\ \dotted{\sddef{\t{r}'}}_{\tb{}}}
            {\sddef{\t{}}}
  } \\
  \sddef{\funtype{\overvec{\t{j}}\ \dotted{\t{r}'}_{\a{}}}{\t{}}} & = \\
  \multicolumn{3}{l}{\hspace*{2em}
    \funtype{\overvec{\sddef{\t{j}}}\ \dotted{\sddef{\t{r}'}}_{\a{}}}
            {\sddef{\t{}}}
    \hspace*{1.5em}
    \text{where $\a{} \neq \a{r}$}
  } \\
  \sddef{\alltype{\overvec{\a{j}}\ \dotted{\a{}}}{\t{}}} & = &
     \alltype{\overvec{\a{j}}\ \dotted{\a{}}}{\sddef{\t{}}} \\
\end{array}$$

$$\begin{array}{l}
  \tddef{\funtype{\overvec[n]{\t{j}}\ \dotted{\t{r}}_{\tb{}}}{\t{}}} \hspace*{1ex} = \\
  \hspace*{2em}
     \funtype{\overvec[n]{\tddef{\t{j}}}\ \overvec[m]{\tddef{\t{r}}[\tb{} \mapsto \tb{k}]}}
             {\tddef{\t{}}}  \\
  \tddef{\funtype{\overvec[n]{\t{j}}\ \dotted{\t{r}}_{\a{}}}{\t{}}} \hspace*{1ex}  = \\
  \hspace*{2em}
     \funtype{\overvec[n]{\tddef{\t{j}}}\ \dotted{\tddef{\t{r}}}_{\a{}}}
             {\tddef{\t{}}}
  \hspace*{2em}
     \text{where $\a{} \neq \tb{}$} \\
\end{array}$$

\caption{Subst-dots and trans-dots\label{fig:metafuns}}
\end{figure}

\end{schemeregion}
