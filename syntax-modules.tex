\begin{schemeregion}
\section{Modules, or You Want it When, Again?}
\label{sect:modules}

The PLT Scheme module system~\cite{f:modules} allows programmers to
group definitions, use imports and exports to control the
scope of names, and specify the dependencies between modules. The
presence of macros complicates the notion of dependence between
modules.

In the presence of procedural macros, a compiler must execute parts of
a program in order to deal with the remainder of the program. This
blurs the line between compilation and execution.
%
In particular, an interpreter may draw the line in a different place
than the compiler, requiring programmers to debug their compiled
program after they have already debugged their interpreted program. To
eliminate this potential for inconsistency, the PLT Scheme module
system require explicit module dependencies and, based on these, 
provides uniform behavior in both interactive and
batch-compilation mode.

% Modules are compilation units, and every module must be compiled
% before it can be used. Modules contain declarations of their direct
% dependencies. When a module is compiled, the module system uses those
% declarations to determine the portions of existing modules that must
% be executed to support the compilation of the current module. 
% %
% If a macro transformer depends on a value definition, the macro's
% module must declare a ``for-syntax'' dependency on the value
% definition's module.
% %
% Scoping rules prevent access from macros to undeclared run-time
% dependencies, and the compiler creates separate instantiations of
% declared dependencies to prevent interference across separate
% compilations.

\subsection{Split environments}

Syntactically, a module declaration contains a
module reference specifying the language that the module is written
in, the module's name, and a sequence of definitions and expressions. 
In our examples, the module's name is left implicit, and provided in
a comment. In the PLT Scheme implementation, the name is taken from
the filename.
\begin{schemedisplay}
hlang initial-language ;; module-name
module-contents $\cdots$
\end{schemedisplay}
Denotationally, a module consists of two code parts (plus a dependency
specification): a compile-time
component and a run-time component. The compile-time part consists of
the syntax definitions. The run-time part consists of ordinary
definitions and expressions. 

The compiler keeps separate environments for the compile-time
expressions and run-time expressions. If a module defines a procedure
as a run-time value, a macro transformer in the same module cannot
\emph{use} that procedure; the binding is unavailable in the
compile-time phase. The macro can, of course, expand into code that
\emph{refers} to the procedure.  Likewise, a binding in the
compile-time phase cannot be used in the run-time phase. This phase
separation permits the compiler to compile a module without also
executing its entire contents.%
\footnote{The same name may have (possibly distinct) meanings in both phases
 simultaneously. For example, modules written in the \scheme{scheme}
 language automatically import all primitive bindings into both
 phases.}

The two environments yield two kinds of module dependencies and thus two
distinct module import forms. The plain \scheme{require} form imports
bindings into the  environment for run-time expressions, and the
\scheme{for-syntax} variant imports bindings into the
environment for compile-time expressions.

\begin{figure}
\begin{schemedisplay}
langs ;; macro-util
(provide check-for-duplicate-identifier)
(define (check-for-duplicate-identifier ids) ELIDED)

langs ;; rec
(require (for-syntax macro-util))
(define-syntax (recur stx)
  (syntax-case stx ()
    [(recur name ([var init] ...) . body)
     (begin
       (check-for-duplicate-identifier #'(var ...))
       #'(letrec ([name (lambda (var ...) . body)])
	   (name init ...)))]))
(define (build-list n f)
  (recur loop ([i 0])
    (if (< i n)
	(cons (f i) (loop (+ i 1)))
	null)))
\end{schemedisplay}
\caption{Four kinds of references}
\label{fig:four-references}
\end{figure}

Macros bridge the gap between the two phases. The implementation of a
macro is a compile-time expression, but the macro definition extends
the environment for run-time expressions. To understand this idea, it
is important to distinguish between the notions of macro versus value
bindings from the notions of environments for compile-time versus run-time
expressions.

The modules in Figure~\ref{fig:four-references} illustrate the four
different possibilities.
%
In the context of the \scheme{rec} module,
\scheme{check-for-duplicate-identifier} is a value binding in the
compile-time environment; thus, it is available for use in the body of
the \scheme{recur} macro definition. Even though
\scheme{check-duplicate-identifier} is a ``compile-time procedure,''
it is not a macro. In fact, it cannot be used in run-time expressions
at all.
%
In contrast, \scheme{recur} is a macro binding in the run-time
environment. It is bound to a compile-time value, but the
binding is available to run-time expressions such as the definition of
\scheme{build-list}.
%
The occurrence of \scheme{syntax-case} refers to a macro binding in
the compile-time environment. Finally, the definition of
\scheme{build-list} creates a value binding in the run-time
environment.

Compilation of a module involves executing its dependencies%
\footnote{If the module depends on modules that are not already
  compiled, they are automatically compiled when the dependency is
  detected.} and expanding uses of macros in the module's body. The
dependencies include the compile-time part of the module's initial
language module, the compile-time part of every module imported with
\scheme{require}, and both compile-time and run-time parts of every
module imported with \scheme{for-syntax} inside \scheme|require|. 

The rules for compilation (and also for invoking a module's compile
time part) are as follows:
\begin{itemize}
\item
For every \scheme{require} import, including the initial language
module, invoke that module's compile-time part in the same phase.
\item
For every \scheme{for-syntax} import, invoke that module's
compile-time and run-time parts in the next higher phase.
\end{itemize}
If a module is imported twice, once with plain \scheme{require} and once with
\scheme{for-syntax}, the two corresponding invocations of the
module are separate. They do not share mutable state. The module
system uses phase numbers to distinguish the different instances.
%
Finally, a module is invoked only once per phase, per
compilation. Multiple modules that depend on a single module in the
same phase share a single invocation of that module and its state.

\subsection{Compilation independence}

True separate compilation is impossible in a module system that
supports the import and export of macros. Instead, the module system
has a principle of compilation independence:
\begin{quote}
Compiling a module depends only on the compiled forms of the
modules that it (transitively) requires.
\end{quote}
This principle has two consequences:
\begin{itemize}
\item The compilation of two modules, neither of which transitively
  requires the other, should produce the same two results no matter
  which is compiled first, or whether they are compiled in parallel.
\item The compilation of a module does not depend on side effects that
  occurred during the compilation of modules that it transitively
  requires. This has important implications for the use of
  side-effects at compile time.
\end{itemize}

The compiler effectively creates a new store for each module
that it compiles. Each compilation gets a new execution of all
supporting module code.
%
Since the result of the compilation process is nothing but a body of
code, the states of mutable variables and objects created during the
compilation process of any module are discarded at the end.

The pair of modules in figure~\ref{fig:side-effect}
 illustrates the interaction between
side-effects and compilation.
\begin{figure}
\begin{schemedisplay}
langs ;; storage
(define storage '())
(define (add! x) (set! storage (cons x storage)))
(provide storage add!)

langs ;; memory
(require (for-syntax storage))
(define-syntax (remember stx)
  (syntax-case stx ()
    [(remember sym)
     (begin (add! (syntax->datum #'sym))
	    (with-syntax ([syms storage])
              #`(begin (display (quote syms))
		       (newline))))]))
(remember a)
(remember b)
\end{schemedisplay}
\caption{Side Effects and Compilation}
\label{fig:side-effect}
\end{figure}
The first module defines two variables. The second module accesses the
variables at compile time, so it imports the first module via
\scheme{for-syntax}. It defines a \scheme{remember} macro that
adds a symbol to the remembered list and generates code to print out
the updated list of remembered symbols. Then it uses the macro
twice. At the end of compiling the \scheme{memory} module, the
\scheme{storage} variable has the value \schemeresult{(b a)}. Executing the
\scheme{memory} module prints out the lists \schemeresult{(a)} and
\schemeresult{(b a)}, as expected.

Consider the following addition to the program:
\begin{schemedisplay}
langs ;; inspect-storage
(require storage)
(require memory)
(display storage) (newline)
\end{schemedisplay}
When this module is executed, the last line it prints out is
\schemeresult{()}, not \schemeresult{(b a)}, because the
\emph{run-time} instance of the \scheme{storage} module is distinct
from the \emph{compile-time} instance. That is, side-effects do not
cross phases.

Now consider this further addition to the program:
\begin{schemedisplay}
langs ;; more
(require memory)
(remember c)
\end{schemedisplay}
When this module is executed, the last line it prints out is
\schemeresult{(c)}, not \schemeresult{(c b a)}.
%
% This result often surprises macro programmers. Many of them expect the
% final line to be \schemeresult{(c b a)}. It seems to them as if the
% effects in \scheme{memory} occur and are subsequently unwound behind
% their backs. Programming with compile-time side effects can result in
% unexpected behavior---or lack of behavior---unless programmers
% recognize the forgetful nature of the compilation process.
%
The reason that the \scheme{(remember c)} in \scheme{more} prints just 
\schemeresult{(c)} is that \scheme{more} was compiled with a fresh
instance of \scheme{storage} (initially the empty list), and because
executing the compile-time part of \scheme{memory} does not change
that value. The variable is updated during \emph{macro expansion};
the side-effects are not present in the compiled form of
\scheme{memory}:
\begin{schemedisplay}
(compiled-module memory
  (require scheme)
  (require (for-syntax storage))
  (define-syntax (remember stx) ELIDED)
  (begin (display '(a)) (newline))
  (begin (display '(b a)) (newline)))
\end{schemedisplay}
\begin{schemeregion}
{
\newcommand\modulebox[4]{%
\hbox{\framebox{\parbox[t][0.92in][t]{1.0in}{\raggedright{\large #1}\\{\small phase #2}\\{\tiny #4}}}}}

%% \newcommand\modulebox[4]{%
%% \hbox{\framebox{\vbox{\hbox to 1.6in{\large #1}\hbox{\small phase #2 of #3}\hbox{\footnotesize #4}}}}}

\newcommand\modline{\hbox{\hskip 0.5in\vrule height0.1in width1pt}}

\newcommand\cmodulebox[4]{\modulebox{#1}{#2}{}{#4}}
\newcommand\rmodulebox[3]{\modulebox{#1}{#2}{}{#3}}

% parts macros
\newcommand\CT{invoke compile-time part}
\newcommand\EandCT{expand macros, invoke compile-time parts}
\newcommand\CTandRT{invoke compile-time and run-time parts}

\begin{figure}
$$
\begin{array}{cccc}
\mbox{compiling \variablefont{storage}} & \mbox{compiling
  \variablefont{memory}} & 
\mbox{compiling \variablefont{more}} & \mbox{executing
  \variablefont{more}} \\ \hline 
\vbox{%
\cmodulebox{storage}{0}{storage}{\EandCT}%
} &
\vbox{%
\cmodulebox{storage}{1}{memory}{\CTandRT}%
\modline%
\cmodulebox{memory}{0}{memory}{\EandCT}%
} & 
\vbox{%
\vspace{2mm}
\cmodulebox{storage}{1}{more}{\CTandRT}%
\modline%
\cmodulebox{memory}{0}{more}{\CT}%
\modline%
\cmodulebox{more}{0}{more}{\EandCT}%
} &
\vbox{%
\rmodulebox{storage}{1}{\CTandRT}%
\modline%
\rmodulebox{memory}{0}{\CTandRT}%
\modline%
\rmodulebox{more}{0}{\CTandRT}%
} \\
\end{array}
$$
\caption{Module invocations for the execution of \variablefont{more}}%
\label{fig:module-invocations}%
\end{figure}
}
\end{schemeregion}

Figure~\ref{fig:module-invocations} shows all of the module invocations
involved in compiling and executing the program \scheme|more|. Each box represents
a module invocation, and the text at the bottom of each box indicates
what parts of the module are executed. Each column represents a shared
store; effects in one column are not visible in another column.

The furthest left column simply represents the compilation of
\scheme|storage|---this module has no \scheme|for-syntax|
dependencies, and so its compilation triggers no computation in other
modules.  The second column is the compilation of \scheme|memory|,
which requires first running the compile-time portions of the
\scheme|storage| module, since \scheme|memory| requires
\scheme|storage| \scheme|for-syntax|,
then expanding any macros in the \scheme|memory| module. The first two
columns are performed since \scheme|storage| and \scheme|memory| are
both dependencies of \scheme|more|.
Third, the \scheme|more| module is compiled.  This requires running
the compile-time portion of \scheme|memory| (which is
\scheme|require|d by \scheme|more|) and therefore the compile- and run-time
portions of \scheme|storage| (which is \scheme|require|d
\scheme|for-syntax| by \scheme|memory|).  Finally, the fourth column
is the final runtime, which invokes both the compile- and run-time
portions of \scheme|more| and \scheme|memory|, as well as
\scheme|storage|.  

\subsection{Persistent effects}
\label{sect:syntax:persistent}

The compilation rules of the module system require the development of
 a design pattern for expressing persistent effects.
%
Since compile-time side effects are transient, only the code in the compiled
module is permanent. Thus, the way to express a persistent effect is
to make it part of the module:
\begin{schemedisplay}
langs ;; memory.v2
(require (for-syntax storage))
(define-syntax (storage-now stx)
  (syntax-case stx ()
    [(storage-here)
     (with-syntax ([syms storage])
       #'(quote syms))]))
(define-syntax (remember stx)
  (syntax-case stx ()
    [(remember sym)
     #'(begin (define-syntax _ (add! (quote sym)))
	      (display (storage-now))
	      (newline))]))
(remember a)
(remember b)
\end{schemedisplay}
The effect of adding new symbols to the \scheme|storage| variable
 is not executed within the macro, but the macro expander
executes the resulting \scheme{define-syntax} form when it continues
expanding the module body, so the effect of the first addition to the
list still occurs before the second \scheme{remember} is
expanded. This version introduces a helper macro,
\scheme{storage-now}, to retrieve the value of \scheme{storage} after
the update.

Since the compile-time part of a compiled module includes all of the
macro definitions, the side-effect is preserved:
\begin{schemedisplay}
(compiled-module memory.v2
  (require scheme)
  (require (for-syntax storage))
  (define-syntax (storage-now stx) ELIDED)
  (define-syntax (remember stx) ELIDED)
  (define-syntax _1 (add! 'a))
  (display '(a)) (newline)
  (define-syntax _2 (add! 'b))
  (display '(b a)) (newline))
\end{schemedisplay}
The calls to \scheme{add!} are executed whenever
\scheme{memory} is required for the compilation of another
module. Thus they are executed when \scheme{more.v2} is compiled
(refer back to Figure~\ref{fig:module-invocations}), so the storage is
already set to \scheme{(b a)} when the use of \scheme{remember} in
\scheme{more} is expanded. Thus, executing the new version of
the program prints \scheme{(c b a)}.

As a matter of readability, the \scheme{begin-for-syntax} form
accomplishes the same effect as the awkward use of
\scheme{define-syntax} with a throw-away name. Using
\scheme{begin-for-syntax} also explicitly signals the programmer's
intent to generate an expression that creates a persistent effect.

%In summary:
%\begin{itemize}
%\item Persistent effects must be part of the compiled module.
%\item \scheme{require} forms indicate dependence, not just the import
%  of names.
%\item There is no way for one module to affect another module that
%  doesn't depend on it.
%\end{itemize}
\end{schemeregion}
