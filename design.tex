\begin{schemeregion}
In this chapter, I describe several key design decisions and their
rationale. 

\section{Reject Ill-typed Programs}

Most previous approaches to static checking for untyped languages have
not rejected programs.  Instead, they validated all programs, and
either issued warnings or added new runtime checks to programs that
could not be proved safe.  In contrast, Typed
Scheme issues an error at compile-time for programs that do not
typecheck.  

This simplifies the design and implementation of Typed Scheme in a
number of ways.  First, Typed Scheme does not have to transform the
program to insert dynamic checks.  Second, it allows the Typed Scheme
type system to be more expressive than the dynamic checks that can be
easily implemented with the PLT Scheme contract system.
  For example, dynamic checks for types with filters
is an open problem.  Third, it makes the programming model much
simpler for users, by comparison to systems which automatically insert
dynamic checks.  Almost all users of Typed Scheme have used typed
languages before, and thus no explanation is required, and the
program's behavior
doesn't change based on the types. Fourth, and most importantly, all
the errors detected by the type system are reported, meaning that the
the programmer can rely on types in the program  for
reasoning about the program.  Even in sound external tools, the
program can still be run when ill-typed, whereas Typed Scheme
statically rejects such programs as a part of the PLT Scheme execution
process. 

\section{Explicit Typing}

With few exceptions, every bound variable in Typed Scheme must be
annotated with its type.  This makes programming more inconvenient by
comparison either to idiomatic untyped Scheme programming, or to other
typed functional languages, especially when anonymous functions are
used.  But it avoids several problematic aspects of type inference.

First, complete inference for the Typed Scheme type system is very
complex, since it combines polymorphism and subtyping.  Second,
type inference is well-known to cause hard-to-decipher type errors
with non-local behavior.  Third, experience with systems that
performed complete type inference for untyped Scheme code suggested
that such systems are extremely brittle, with minor code changes
radically changing the inference results.  Fourth, type inference
allows programmers to omit type annotations in many cases where the
type would serve as valuable statically-checked documentation.

Instead, Typed Scheme provides local type inference in certain cases:
for type arguments to polymorphic functions, for non-recursive local
bindings, and for some $\lambda$-bound variables.  This local
inference removes much of the most significant burden from the
programmer, while simplifying the implementation and preserving
valuable checked documentation.

\section{Module-level Granularity}

When using Typed Scheme, a module must be either wholly typed, or
wholly untyped.  In contrast, many other approaches to ``gradual
typing'' have allowed mixing typed and untyped code at the expression
level.  

Both approaches obviously have the same expressiveness, since
any piece of code can be factored out into its own module.  However,
the module-by-module approach has several advantages.  First, modules
are a natural level to think about the organization of code.  Second, 
since the contracts generated at typed/untyped boundaries have runtime
cost, module level granularity makes reasoning about the performance
impact of typed/untyped boundaries easier.  Third, the PLT module
system allows individual modules to specify their language, making the
module level the most natural granularity for integration with the
rest of PLT.  

\section{Pre-expanding Macros}

In Typed Scheme, all macros in the source program are expanded before
typechecking.\footnote{This is also the approach taken by the ACL2
  theorem prover~\cite{acl2-book}.}  This has several vital
advantages.
  First, since untyped Scheme code
may use arbitrary macros, which do not come pre-equipped with type
rules, this allows Typed Scheme to handle the full range of existing
programs.  Second, it allows the implementation to deal with just the few
core forms of PLT Scheme.  

Since almost all control structures in PLT Scheme are defined in terms
of \scheme|if|, occurrence typing  works for them without
additional modification.  For example, the \scheme|match| library
provides a sophisticated pattern matcher, which compiles to plain PLT
Scheme.  Typed Scheme works properly for almost all uses of
\scheme|match|, without any special implementation effort. Since PLT
programmers use thousands of different macros, this is the only way to
make the implementation manageable, as well as adaptable to new
macros. 

However, some macros introduce invariants that must be understood by
the type system for proper typechecking.  These must be handled
specially.  For example, the \scheme|define-struct| macro must
communicate with the typechecker to introduce new types, and thus
Typed Scheme introduces the new \scheme|define-struct:|.  This means
that further complex macros of this sort require special handling,
preventing their use in Typed Scheme currently.  The problem of sound
and simple macro-extensibility of typed languages is still open.  

\section{No New Idioms}

For Typed Scheme, I took the idioms and styles of PLT Scheme
programmers as they exist.  The addition of a type system allows for
many new styles of programming, and a type system designer is always
tempted to add these.  I have attempted to resist this temptation, and
only add features to typecheck PLT Scheme programs that already
exist.\footnote{Refinement types are somewhat of an exception to this
  rule.}  This has provided a rigorous bound on what features to add
to Typed Scheme, and on what the goals of the Typed Scheme project
are.  As Typed Scheme becomes more widely used, and is used as the
initial language for programs, this decision may need to be revisited.
\end{schemeregion}
