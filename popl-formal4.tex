\begin{schemeregion}

\section{Proving Soundness}
\label{sec:soundness}

Next we turn our attention to the type soundness of \lts.
Two issues deserve consideration.

\subsection{Typing Intermediate Steps}
\label{sec:intermed}
The system as described does not obey the usual subject reduction
property.\footnote{This problem, with the same solution, is also
  described in the original paper on Typed Scheme~\cite{thf:popl08}.} 
 For example, the following program

\begin{schemedisplay}
  ((lambda ([x : Any])
     (if (number? x)
	 (add1 x)
	 0))
   #f)
\end{schemedisplay}

\noindent
reduces in one step to 

\begin{schemedisplay}
  (if (number? #f)
      (add1 #f)
      0)
\end{schemedisplay}

\noindent
which does not typecheck under the typing rules for \lts, since
\scheme|add1| requires a numeric argument.  Of course, this
untypeable code is never executed, so the program is actually safe.  
%
To solve this problem, we add two new typing rules; see
figure~\ref{fig:ptrules}.  

\begin{figure}
  \[
\inferrule[T-IfTrue]
{
  \hastyeff{\e1}{\t1}{\empt}{\bot}{\s1}
  \\\\
  \hastyeffphi{\e2}{\t2}{\phii2} {\s2}
  \\\\
  \subtype{\t2}{\t{}} \\\\
  \phii{} = \combfilter{\empt|\bot}{\phii2}{\empt|\empt}
}
{\hastyeffphi {\cond{\e1}{\e2}{\e3}} {\t{}}  {\phii{}} {\noeffect}}
\quad
\inferrule[T-IfFalse]
{
  \hastyeff{\e1}{\t1}{\bot}{\empt}{\s1}
  \\\\
  \hastyeffphi{\e3}{\t3}{\phii3} {\s3}
  \\\\
  \subtype{\t3}{\t{}} \\\\
  \phii{} = \combfilter{\bot|\empt}{\empt|\empt}{\phii3}
}
{\hastyeffphi {\cond{\e1}{\e2}{\e3}} {\t{}}  {\phii{}} {\noeffect}}
\]

  \caption{Auxiliary Type Rules for Subject Reduction Proof}
  \label{fig:ptrules}
\end{figure}

Since  there is no overlap between \num and \fft, we have that
$$\hastyeffphi{\comb{\numberp}{\ff}}{\bool}{\bot|\empt}{\noeffect}$$
Therefore {\sc T-IfFalse} applies to our
problematic example, allowing the type system to ignore the troublesome 
\tbranch. With these additional rules, we can proceed to proving the usual 
subject reduction theorem.

Of course, the original system is still sound.  We can also prove that
these rules are unnecessary.  Any program which typechecks without
the new rules also typechecks with them.  Therefore, the system
without the new rules is sound, since by the soundness theorem for the
extended system, no runtime errors are possible.  

Further, the existence of the {\sc T-IfTrue} and {\sc T-IfFalse} rules
demonstrate that Typed Scheme can, in many instances, detect dead code
in the program.  In many cases, this code is the result of macro
expansion, but in the case where user-written code is provably
unreachable, Typed Scheme issues a warning.  

To distinguish uses of these two rules, we will write the judgement
 $$\hastyeffo{\e{}}{\t{}}{\ph{+}}{\ph{-}}{\s{}}$$ with the use of
 $\vdash_o$ to indicate that the rules {\sc T-IfFalse} and {\sc
   T-IfTrue} were not used in the derivation.\footnote{$o$ is a
   mnemonic for ``original''.}

\subsection{Relating Substitutions and Filters}

\def\tfs{\ma{\empt|\bot}}
\def\ffs{\ma{\bot|\empt}}

As usual, proving subject reduction for \lts requires a lemma
showing that  substitution preserves types.  However, this is non-trivial for \lts, since the
filter set may change in significant ways due to a
substitution. Consider the example from section~\ref{sec:intermed}. The filter set for
\scheme|(number? x)| where \scheme|x| has type $\top$ is
$\num_{\x{}}|\comp{\num}_{\x{}}$.  If we apply the
substitution $[0/\x{}]$, the resulting expression
has filter set \tfs. If we instead apply the substitution
$[\ff/\x{}]$, we get the filter set \ffs.  These resulting filter
 sets do not have a simple relationship either to each other or 
to the original filter set.  Therefore, we must have a
way of incorporating the substitution into the relation between filters as well.

\begin{figure}
  \[
\inferrule{
 \forall \x{} \in \mbox{dom}(\G{}) \subtype{(\G{} + \opp{})(x)}{(\G{} + \op{})(x)}
}
{
  \G{} \vdash \op{} < \opp{}
}
\]
\vspace{.2in}
\[
\hastype{\x{}}{\t{}} \models \bot
\qquad
\inferrule{\subtype{\t{}' @ \pi}{\t{}}}
{\hastype{\x{}}{\t{}'} \models \t{\wpi{\x{}}}}
\qquad
\inferrule{\notsubtype{\t{}' @ \pi}{\t{}}}
{\hastype{\x{}}{\t{}'} \models \compt{\wpi{\x{}}}}
\]
\[
\inferrule{\x{} \neq \y{}}
{\hastype{\x{}}{\t{}'} \models \t{\wpi{\y{}}}}
\quad
\inferrule{\x{} \neq \y{}}
{\hastype{\x{}}{\t{}'} \models \compt{\wpi{\y{}}}}
\quad
\inferrule{\hastype{\x{}}{\t{}} \models \p{} \text{ or }
  \hastype{\x{}}{\t{}'} \not\models \p{}'}
{\hastype{\x{}}{\t{}'} \models \imp{\p{}'}{\p{}}}
\]

  \caption{Relations on Filters}
  \label{fig:filters}
\end{figure}


To describe this relation, first we define a ``more specific than''
relation on sequences of filters, $\G{} \vdash \op1 < \op2$ (with \op1 more specific),
; see
figure~\ref{fig:filters}.  This relation captures the intuition that a
sequence of filters is more specific if it allows more terms to be
typed in a given environment.  

Second, we define a ``models'' relation that relates pairs of variables and types
 to
filters.  A variable with a type models a filter if it agrees with the types
specified by the filter for the relevant variables.  Thus 
$\x{} : \num \models \num_{\x{}}$,  $\x{} : \fft \not\models
\num_{\x{}}$,  and $\x{} : \fft \models
\comp{\num}_{\x{}}$.  Using this relation, we can see that if we
substitute a value \vv{} of type \t{} for a variable \x{} in
expression \e{}, one half of the filter set of the result is modeled by $\x{} :
\t{}$, and one is not.  Furthermore, the filter set of the resulting
term is more specific in the environment without \x{}.

Equipped with this additional machinery, we can state the central lemma 
about substitution for the subject reduction proof.
%\newpage
\begin{lemma}
\label{lem:subst}
  If \hastyeffphi [\Delta,\G{},{\movervec[i]{\hastype{\x{}}{\sig{}}}}] {\e{}} {\t{}} {\op{+} | \op{-}} {\s{}}

and \hastyeffphi [] {\vv{}} {\sig{}'} {\phii{0}} {\s{0}}

and \subtype{\sig{}'}{\sig{}}

\noindent
Then $\hastyeffphi {\subs{\e{}} {\vv{}} {\x{}}} {\t{}'} {\opp{+} | \opp{-}} {\s{}'}$

and \subtype{\t{}'}{\t{}}

and $\hastype{\x{}}{\sig{}'} \models \op{+} \quad \Rightarrow \quad \G{} \vdash \opp{+} < \op{+}$

and $\hastype{\x{}}{\sig{}'} \models \op{-} \quad \Rightarrow \quad \G{} \vdash \opp{-} < \op{-}$

and $\hastype{\x{}}{\sig{}'} \not\models \op{+} \quad \Rightarrow \quad \opp{+} = \bot$

and $\hastype{\x{}}{\sig{}'} \not\models \op{-} \quad \Rightarrow \quad \opp{-} = \bot$

and either $\s{} = \wpi{\x{}} \wedge \s{}' = \noeffect$ or $\s{} = \s{}'$

\end{lemma}

\noindent\textbf{Proof Sketch}
This lemma is proved by induction on the original type derivation. 

The interesting cases are for the rules for \scheme|if|. In this case,
we consider the four implications regarding whether the actual type of
the value $v$ ($\sigma'$) makes the predicate true or false, or
whether the result is still indeterminate.  If the test implies that
the then branch is never reached after the substitution, then the
resulting filter will be $\bot$, and {\sc T-IfFalse} can be used to
type the result.  Otherwise, the resulting filters are at least as
strong, and the previous proof of the type of the then branch will
still work.  The case for the else branch is symmetric. \qed

Given this lemma, we can prove the usual subject reduction and
progress theorems in the style of \citet{wf:type-soundness}, and then
prove the irrelevance of the additional type rules.

\begin{lemma}[Preservation]
  If \hastyeffphi[\ ]{e}{t}{\op{+}|\op{-}}{\noeffect} and $e$ is closed
  and \reduce{e}{e'}, then
  \hastyeffphi[\ ]{e'}{t'}{\opp{+}|\opp{-}}{\noeffect} where
  \subtype{t'}{t} and $\vdash \opp{+} < \op{+}$ and $\vdash \opp{-} <
  \op{-}$ 
  and $e'$ is closed.
\end{lemma}

Note that we need not consider open terms
for the purpose of proving soundness.  Further, only open terms may
have filters which reference variable, or objects.  Therefore,
$\op{+}|\op{-}$ is either $\empt|\empt$, $\empt|\bot$ or $\bot|\empt$ 

\noindent\textbf{Proof Sketch}
The proof is by cases on the reduction rule used to derive
\step{e}{e'}.
In almost all of the cases, the proof is trivial.  The most
interesting case is for the rule {\sc E-Beta}, where
lemma~\ref{lem:subst} is used.  For the rules {\sc E-IfTrue} and {\sc
  E-IfFalse}, note that every value either has filter $\empt|\bot$ or
$\bot|\empt$. For rule {\sc E-Delta}, we simply need to consider cases
on which constant is the operand.
\qed

\begin{lemma}[Progress]
  If \hastyeffphi[\ ]{e}{t}{\op{+}|\op{-}}{\noeffect} and $e$ is closed
  then \reduce{e}{e'} or $e$ is a value.
\end{lemma}
\noindent\textbf{Proof Sketch}
By straightforward induction on the derivation of\\
\hastyeffphi[]{e}{t}{\op{+}|\op{-}}{\noeffect}.
\qed

\begin{lemma}[Soundness]
\label{lem:sound}
  If \hastyeffphi[\ ]{e}{t}{\op{+}|\op{-}}{\noeffect} and $e$ is closed
  then there is some \vv{} such that \reduces{e}{v} and 
  \hastyeffphi[\ ]{v}{t'}{\opp{+}|\opp{-}}{\noeffect} where
  \subtype{t'}{t} and $\vdash \opp{+} < \op{+}$ and $\vdash \opp{-} <
  \op{-}$ or for all
  $\e{}'$ where \reduces{\e{}}{\e{}'}, there exists $e''$ such that
  $\reduce{e'}{e''}$.
\end{lemma}
\noindent\textbf{Proof Sketch}
By the Wright-Felleisen method.
\qed

Given this theorem, we can conclude soundness for the original system
at base types (it is stated here for number for conciseness).
  Note that this is not true for higher types.  For
example, this program:

\begin{schemedisplay}
((lambda: ([x : Any]) 
   (lambda: ([y : Number]) 
     (if (number? x)
	 (add1 x)
	 0)))
 #t)
\end{schemedisplay}
\noindent
is typable in the original system, but it reduces to a value in one
step that is typable only in the extended system.

\begin{theorem}[Soundness for the original system]
 ~\newline  If \hastyeffphio[\ ]{e}{\num}{\op{+}|\op{-}}{\noeffect} and $e$ is closed,
  then there is some \vv{} such that \reduces{e}{v} and 
  \hastyeffphio[\ ]{v}{\num}{\empt|\bot}{\noeffect} or for all
  $\e{}'$ where \reduces{\e{}}{\e{}'}, there exists $e''$ such that
  $\reduce{e'}{e''}$.
\end{theorem}
\noindent\textbf{Proof Sketch}
First, it is obvious that if
 \hastyeffphio[\ ]{e}{\num}{\op{+}|\op{-}}{\noeffect}, then
 \hastyeffphi[\ ]{e}{\num}{\op{+}|\op{-}}{\noeffect}.  Therefore, from
 lemma~\ref{lem:sound} we have that $\reduces{e}{\vv{}}$ where
\hastyeffphi[\ ]{v}{t'}{\empt|\bot}{\noeffect} and
  \subtype{t'}{\num}, unless \e{} reduces infinitely. 
  Any value whose type is a subtype of
  \num must be a number, thus \vv{} is a number.   Therefore, 
\hastyeffphio[\ ]{v}{\num}{\empt|\bot}{\noeffect}, by rule {\sc T-Num}.  \qed

\end{schemeregion}
